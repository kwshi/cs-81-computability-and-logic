\documentclass{cs81-homework}

\title{Assignment 5}
\author{}
\date{2018 February 26 (Monday)}

\begin{document}

\begin{introduction}
  \theintroduction
\end{introduction}

\begin{enumerate}

\item[] These problems are formal proofs about the natural numbers. Use natural
  deduction and the axioms of Peano Arithmetic (PA) from class, including
  induction where needed.  Here \(0\) is the constant symbol and \(1\) is
  defined as \(0'\), where \('\) is the successor function.

  You may use any results proved in the class slides, but you should cite
  anything you use by label if possible.  Don't forget to use the \(\forall\)
  Elimination rule when you plug into those axioms and lemmas.  Cite what you
  are plugging in, e.g. \sub{a+b}{x} means \(a+b\) is being substituted for
  \(x\).
  
\item \points{5} Formally prove, using natural deduction,
  \[
    \forall y \: y + 1 = y'.
  \]
    
  \begin{solution}
  \end{solution}

\item \points{5} Formally prove, using natural deduction,
  \[
    \forall y \: 1 \times y = y.
  \]
  
  \begin{solution}
  \end{solution}

\item \points{10} Formally prove, using natural deduction,
  \[
    \forall y \: y \times 1 = y.
  \]

  \begin{solution}
  \end{solution}

\item \points{5} (Basis only) Formally prove, using natural deduction, that
  \(\times\) distributes over \(+\), i.e.
  \[
    \forall x \: \forall y \: \forall z \: x \times (y + z) = (x \times y) + (x
    \times z).
  \]
  You may use any results proved in the slides or in previous problems as
  lemmas.  For this problem, induction is required.  So you will be showing
  \[
    \forall x \: A(x) \quad \text{where} \quad A(x)\colon \forall y \: \forall z
    \: x \times (y + z) = (x \times y) + (x \times z).
  \]
  In this problem, only show the basis of the proof.  The induction step is in
  the next problem.
  \[
    A(0)\colon \forall y \: \forall z \: 0 \times (y + z) = (0 \times y) + (0
    \times z).
  \]

  \begin{solution}
  \end{solution}

\item \points{25} (Induction step only) Formally prove, using natural deduction,
  that \(\times\) distributes over \(+\), i.e.
  \[
    \forall x \: \forall y \: \forall z \: x \times (y + z) = (x \times y) + (x \times z).
  \]
  You may use any results proved in the slides or in previous problems as lemmas.
  For this problem, induction is required.  So you will be showing
  \[
    \forall x \: A(x) \quad \text{where} \quad A(x)\colon \forall y \: \forall z
    \: x \times (y + z) = (x \times y) + (x \times z).
  \]
  In this problem, only show the induction step: \(A(a) \derives A(a')\), i.e.
  \[
    \forall y \: \forall z \: a \times (y + z) = (a \times y) + (a \times z)
    \derives \forall y \: \forall z \: a' \times (y + z) = (a' \times y) + (a'
    \times z).
  \]

  \begin{solution}
  \end{solution}

\item \points{25} One way to introduce a new predicate symbol representing a
  concept into a theory is to treat formulas using the symbol as abbreviations
  for other formulas.  In this problem we define an abbreviation for \(\le\)
  (less-than-or-equal):
  
  For any terms \(s\) and \(t\), \(s \le t\) means \(\exists w \: s+w = t\).
  
  For example \(3+2 \le 4+3\) means \(\exists w \: (3+2)+w = (4+3)\).
        
  Give a natural deduction proof for the following derived inference rule:
  \[
    \prftree[r]{\quad\text{where \(r, s, t\) are any terms.}}{r \le s}{s \le
      t}{r \le t}.
  \]
  (Treat the antecedent formulas as premises, and the consequent as the
  conclusion. Treat \(r\), \(s\), and \(t\) as if constants.)

  \begin{solution}
  \end{solution}

\item \points{25} For the \(\le\) symbol defined in the previous problem, prove
  that \(\le\) is antisymmetric:

  \(\forall x \: \forall y \: ((x \le y) \land (y \le x) \to x = y)\)

  (It might be helpful to invent and prove a lemma.)
  
  \begin{solution}
  \end{solution}

\end{enumerate}
\end{document}