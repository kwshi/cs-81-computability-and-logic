\documentclass{cs81-homework}

\title{Assignment 3}
\author{}
\date{2018 February 12 (Monday)}

\begin{document}

\begin{introduction}
  \theintroduction

  Please help us grade by placing your solutions in the space provided.
\end{introduction}

Translate the following English statements into predicate logic, assuming \(L\)
is a binary predicate representing Likes on a domain of people, \(A\) is a unary
predicate representing whether its argument is an Athlete, and \(M\) is a unary
predicate representing whether its argument is a Musician.  When we speak of
first, second, third person, etc. it is possible that any or all could be the
same person as well; they do not have to be distinct.

\begin{enumerate}
\item \points{3} Everyone likes some athlete and some musician.

  \begin{solution}
  \end{solution}

\item \points{3} There is someone who likes a musician who likes every athlete.

  \begin{solution}
  \end{solution}

\item \points{3} For every pair of people, if the first likes the second, then the
  second likes the first. (This is called the symmetric property.)

  \begin{solution}
  \end{solution}

\item \points{3} For every pair of people, if the first likes the second, then the
  second does not like the first. (This is called the antisymmetric property.)

  \begin{solution}
  \end{solution}

\item \points{3} For any three people, if the first likes the second, and the
  second likes the third, then the first likes the third. (This is called the
  transitive property.)

  \begin{solution}
  \end{solution}

\item[] Translate the following predicate logic statements into english, with the
  same assumptions about L, A, and M as previous. Try to make your statements
  simple and concise.
  
\item \points{3} \(\forall x \exists y (L(x, y) \land \lnot L(y, x))\)

  \begin{solution}
  \end{solution}

\item \points{3} \(\exists x (M(x) \land \forall y (A(y) \to  \lnot L(x, y)))\)

  \begin{solution}
  \end{solution}

\item \points{3} \(\forall x \forall y (L(x, y) \to  \forall z (L(x, z) \to  L(y, z)))\)

  \begin{solution}
  \end{solution}

\item \points{3} \(\forall x \forall y (L(x, y) \to  \exists z (L(x, z) \land L (y, z)))\)

  \begin{solution}
  \end{solution}

\item \points{3} \(\exists x (M(x)  \to  \forall y M(y))\)

  \begin{solution}
  \end{solution}

\item[] Present an informal argument that the following is valid (evaluates to
  true) for any interpretation.
  
\item \points{5} \((\exists x A(x)) \to  ((\forall x M(x)) \to  (\exists x M(x)))\)

  \begin{solution}
  \end{solution}

\item[] Present an informal argument that the following is valid (evaluates to
  true) for any interpretation.

  \begin{solution}
  \end{solution}

\item \points{5} \((\forall x M(x)) \lor (\exists x \lnot M(x))\)

  \begin{solution}
  \end{solution}

\item[] Give an interpretation for which the following is not valid. An
  interpretation with a small domain is preferred. In this counterexample, you
  should show the domain and interpretation of each predicate explicitly, then
  argue that the formula evaluates to false (0).
  
\item \points{5} \((\forall x M(x)) \lor \lnot \exists x M(x)\)

  \begin{solution}
  \end{solution}

\item[] Give an interpretation for which the following is not valid. An
  interpretation with a small domain is preferred. In this counterexample, you
  should show the domain and interpretation of each predicate explicitly, then
  argue that the formula evaluates to false (0).
  
\item \points{5} \((\exists x A(x)) \to  \forall y A(y)\)

  \begin{solution}
  \end{solution}

\item[] Give an interpretation for which the following is not valid. An
  interpretation with a small domain is preferred. In this counterexample, you
  should show the domain and interpretation of each predicate explicitly, then
  argue that the formula evaluates to false (0).

\item \points{10} \(((\forall x A(x)) \to  (\forall x M(x))) \to  \forall x (A(x) \to  M(x))\)

  \begin{solution}
  \end{solution}

\item[] Using JAPE, provide a proof of the following sequent. JAPE requires
  periods after quantified variables. We include them so that you can scrape the
  formula into JAPE directly.
\item \points{15}
  \((\forall x.(A(x) \to B(x))) \to (\exists x.A(x))\to \exists x.B(x)\)
  [constructive proof required.]

  \begin{solution}
  \end{solution}

\item[] Using JAPE, provide a proof of the following sequent. JAPE requires
  periods after quantified variables. We include them so that you can scrape the
  formula into JAPE directly.

\item \points{10} \((\forall x.A(x)) \lor \exists x.\lnot A(x)\)
  
  [This will be a classical proof, somewhat analogous to LEM.]

  \begin{solution}
  \end{solution}

\item[] Using JAPE, provide a proof of the following sequent. JAPE requires
  periods after quantified variables. We include them so that you can scrape the
  formula into JAPE directly.

\item \points{15}
  \((\exists x.A(x)) \to \exists x.(\lnot A(x) \lor \forall y.A(y))\)
  
  [Classical and tricky. I recommend using problem 17 as a lemma.]    

  \begin{solution}
  \end{solution}

\end{enumerate}
\end{document}
