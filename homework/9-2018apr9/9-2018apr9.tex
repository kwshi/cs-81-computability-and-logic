\documentclass{cs81-homework}

\title{Assignment 9}
\date{Monday, April 9 at 6:00 PM}
\author{}

\begin{document}

\begin{introduction}
  \theintroduction
\end{introduction}

\begin{enumerate}
\item \points{10} Let \(N\) be the set of all natural numbers
  \(\set{0, 1, 2, 3, \dots}\).  Which of the following sets are \emph{countable}
  and why?
  \begin{enumerate}[label=\alph*.]
  \item The set \(N \times N\) of all pairs of natural numbers.
  \item The set \(N^\star\) of all finite sequences of natural numbers.
  \item The set of all functions of the form \(N \to N\).
  \item The set of all Turing transducers that compute total functions
    \(\Sigma^\star \to \Sigma^\star\) (with \(\Sigma^\star\) being the set of
    all finite strings of elements in \(\Sigma\) as usual) where the transducers
    are coded in a fixed finite alphabet.  (A \emph{total function} is a partial
    function that is defined for all argument values in \(\Sigma^\star\).  That
    is, it is just a function.)
  \end{enumerate}
  [Hint: Relate the set in question to some other set known to be countable or
  uncountable.]

  \begin{solution}
  \end{solution}

\item \points{10} Similar to the way in which \(\abr M\) represents the code for
  Turing machine \(M\), \(\abr{M, x}\) represents as a single string a
  \textbf{pair} consisting of \(\abr M\) and an input string \(x\) to \(M\).
  For example, we could use a special character, say \(\$\), outside the
  alphabet for either as a divider between \(\abr M\) and \(x\), as in
  \(\abr M \$ x\).  Then we can construct languages with strings representing
  such pairs, for example the language
  \[
    \mathrm{Accepts} = \set{\abr{M, x} \mid \text{\(M\) accepts \(x\)}}.
  \]
  Is \(\mathrm{Accepts}\) decidable, recognizable, corecognizable, or none of
  these?  Prove your answers.  [Note: It is fair to assume that a machine either
  accepts or diverges, as mentioned in the lecture notes.]

  \begin{solution}
  \end{solution}

\item \points{10} Similar to the above, except now we consider triples,
  \(\abr{M, q, x}\) where \(q\) is a specific control state of \(M\).  For
  example, there is a language
  \[
    \mathrm{Reaches} = \set{\abr{M, q, x} \mid \text{\(M\) eventually reaches
        control state \(q\) when started on \(x\)}}.
  \]
  Is \(\mathrm{Reaches}\) decidable, recognizable, corecognizable, or none of
  these?  Prove your answers.  [Hint: Can a machine be modified to have only a
  single accepting state and still recognize the same language?]

  \begin{solution}
  \end{solution}

\item \points{20} For any finite string \(x\), let \(\abs x\) represent the
  length of \(x\).  We define a Turing machine \(M\) to have \emph{prime nature}
  if the only strings \(x\) that \(M\) accepts are such that \(\abs x\) is a
  prime number.  In other words
  \(\set{\abs x \mid x \in L(M)} \subseteq \text{all prime numbers}\).  As
  usual, \(L(M)\) is the language of strings accepted by \(M\).

  An example of a machine with prime nature is one that decides the language
  \[
    \set{\mathtt 1^p \mid \text{\(p\) is a Mersenne prime}} = \set{\mathtt{111},
      \mathtt{1111111}, \dots}.
  \]
  (\(\mathtt 1^p\) here representing \(p\) \(\mathtt 1\)'s in a row).  (A
  \href{https://en.wikipedia.org/wiki/Mersenne_prime}{Mersenne prime} has the
  special form \(2^p-1\) where \(p\) is prime.  It is not known whether or not
  this language is finite, but tests for having this form and for being prime
  exist, so this language is decidable.)

  Show that the language
  \[
    \mathrm{PN} = \set{\abr M \mid \text{\(M\) is a Turing machine having prime
        nature}}
  \]
  is not recognizable.
  
  [Hint: Whether an element \(\abr M\) in \(\mathrm{PN}\) has prime length is
  not relevant.]

  \begin{solution}
  \end{solution}

\item \points{10} Referring to the previous problem, show that the complement
  \(\mathrm{PN}^\complement\) is recognizable.

  \begin{solution}
  \end{solution}

\item \points{5} Define a Turing machine \(M\) to have \emph{infinite nature}
  provide that the language \(M\) recognizes is infinite.  Show that the
  language
  \[
    \mathrm{IN} = \set{\abr M \mid \text{\(M\) is a Turing machine having
        infinite nature}}
  \]
  is not corecognizable.  In other words, show there is not recognizer for
  \[
    \mathrm{IN}^\complement = \set{\abr M \mid \text{\(M\) is a Turing machine
        such that \(L(M)\) is finite}}.
  \]
  Again, \(L(M)\) is the language of strings that \(M\) accepts.

  [Hint: \(\varnothing\) is finite and \(\Sigma^\star\) is infinite.]

  \begin{solution}
  \end{solution}

\item \points{15} Using the definition of \(\mathrm{IN}\) in the previous
  problem, show that
  \[
    \mathrm{IN} = \set{\abr M \mid \text{\(M\) is a Turing machine having
        infinite nature}}
  \]
  is not recognizable.

  [Hint: Note similarity to accepting \emph{all} strings and refer to lecture
  notes.]

  \begin{solution}
  \end{solution}

\item \points{10} We show in the lecture that a language is Computably
  Enumerable iff it is Recognizable.  Define a language to be
  \emph{Monotonically Computably Enumerable} (MCE) provided that the strings of
  \(L\) can be enumerated in increasing order, where order for strings is
  defined as if each string encodes a natural number in the usual fashion.  That
  is, \(L\) is MCE, if there is a Turing transducer computing function \(f\)
  where
  \[
    \forall x \in N \; \forall y \in N \; \del{\text{\(x<y\) implies
        \(f(x) < f(y)\)}}.
  \]
  Show that an infinite language is MCE iff it is decidable.
  
  \begin{solution}
  \end{solution}

\item \points{10} Give an informal (but nonetheless convincing) algorithm for
  deciding the following language:
  \begin{align*}
    \{\abr M \mid\: &\text{\(M\) is a Turing machine that, when started on
                      all-blank tape,} \\
                    & \text{eventually prints something other than a blank}\}.
  \end{align*}
  As usual, \(\abr M\) means a string encoding \(M\). 

  \begin{solution}
  \end{solution}

\end{enumerate}
    
\end{document}