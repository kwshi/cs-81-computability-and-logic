\documentclass{cs81-homework}

\title{Assignment 7}
\date{Monday April 2 at 6:00 PM}
\author{}

\begin{document}

\begin{introduction}
  \theintroduction

  \paragraph{Caution:}
  Do not start on problems 1 and 2 the day of the deadline.  For the first two
  problems, use the Turing machine simulator that I’ve provided on Piazza:
  \href{https://docs.google.com/document/d/1DTeea4vqzkiHAo1imlpkXfG87q42BOmHq-jj4yDWGB0/edit}{tm.pro}
  This is a simulator written in Prolog (\textbf{Pro}gramming in
  \textbf{Log}ic).

  (You can develop in other simulators such as Jflap if you must, but your
  submitted code must still be in our model.)

  There are two ways to run the simulator:
  \begin{enumerate}
  \item Install a Prolog interpreter/compiler on your computer. swipl is a good
    choice. I used it to develop this simulator. I prefer to run it inside a
    text editor (emacs) but it can be run bare, e.g. on the command line.
  \item Use an online version in a browser.
    \url{http://rextester.com/l/prolog_online_compiler} seems like a good
    choice. It is compatible with swipl and is fast.
    (\href{https://swish.swi-prolog.org/p/vWMPonBw.pl}{swish} is not recommended
    as it is slow.)
  \end{enumerate}

\end{introduction}

\begin{enumerate}
\item \points{40} The \(n^\text{th}\) \underline{triangular number} satisfies
  the recurrence
  \begin{align*}
    t(0) &= 0, \\
    t(n) &= n + t(n-1) \text{ for } n > 0.
  \end{align*}
  Thus \(t(0) = 0\), \(t(1) = 1\), \(t(2) = 3\), \(t(3) = 6\), \(t(4) = 10\),
  \(t(5) = 15\), etc.

  Design a Turing transducer \textbf{\texttt{tri1}} that behaves as follows:

  Given an initial tape with \(n\) consecutive \(\mathtt 1\)s,
  \textbf{\texttt{tri1}} modifies the tape to halt with exactly \(t(n)\)
  consecutive \(\mathtt 1\)s.  Thus, if the initial tape is
  \[
    \mathtt{11111}
  \]
  then the final tape is
  \[
    \mathtt{111111111111111}
  \]
  since \(t(5) = 15\).

  Assume the blank symbol is \(\mathtt b\).  You may use other symbols as
  needed.  Assume that the head is intially at the left character of the input.
  In the case of \(n=0\), the entire tape will be blank.

  Submit the code for \textbf{\texttt{tri1}} and a trace of
  \textbf{\texttt{tri1}} running on input \(\mathtt{111}\).  Also show the
  outputs without tracing for inputs 0 to 6 \(\mathtt 1\)s.

  Here is a test setup you can use (\texttt{resultsInRange} was added to
  \texttt{tm.pro} at 2:00 PM on Tuesday, March 27, 2018):
  
  \lstinputlisting{prolog/homework/1-tri1-template.pl}

  \begin{solution}
  \end{solution}

\item \points{30} Design a Turing decider \textbf{\texttt{tri2}} that decides
  the language \(T\) of strings in \(\set 1^\star\) that are triangular numbers.
  That is,
  \[
    T = \set{\varepsilon, \mathtt 1, \mathtt{111}, \mathtt{111111},
      \mathtt{1111111111}, \mathtt{111111111111111}}
  \]
  If an input is an elements of \(T\) then \textbf{\texttt{tri2}} accepts the
  input, otherwise it rejects the input.

  Submit the code for \textbf{\texttt{tri2}}, a trace of \textbf{\texttt{tri2}}
  running on input \(\mathtt{111111111111111}\) and a trace of
  \textbf{\texttt{tri2}} running on input \(\mathtt{1111111111111}\).  Also show
  accepted inputs, without tracing for inputs with lengths in range 0 to 21.

  Here is a testing setup you can use:
  \lstinputlisting{prolog/homework/2-tri2-template.pl}

  \begin{solution}
  \end{solution}

\item \points{15} Decidable languages, i.e. languages for which there is a
  Turing decider, are closed \emph{under intersection}.  This means that, if
  \(L_1\) and \(L_2\) are decidable, then \(L_1 \cap L_2\) is also decidable.
  Describe in sufficient detail to be convincing how to construct a Turing
  decider for \(L_1 \cap L_2\) from Turing deciders for \(L_1\) and \(L_2\).
  Focus on how the rules for the decider for \(L_1 \cap L_2\) would be
  constructed.

  \begin{solution}
  \end{solution}

\item \points{15} Recognizable languages, i.e. languages for which there is a
  Turing recognizer, are closed \emph{under union}. This means that, if \(L_1\)
  and \(L_2\) are recognizable, then \(L_1 \cup L_2\) is also recognizable.  Use
  the Church--Turing thesis to argue why this is true.  Under this thesis, you
  do not have to give the construction details of a Turing machine. Instead you
  can just present a convincing informal algorithm for recognizing
  \(L_1 \cup L_2\).  (Hint: The key issue here is that either of the recognizers
  for \(L_1\) or \(L_2\) could diverge on an input \(x\).  So if you simply run
  the machines in one order or the other, then the first one run could diverge
  while the second one might not.  So it might be concluded that an input string
  x is not in \(L1 \cup L2\) when in fact it is.  So you must describe how to
  protect against such divergence.)

  \begin{solution}
  \end{solution}
\end{enumerate}

  

\end{document}