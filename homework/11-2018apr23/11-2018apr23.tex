\documentclass{cs81-homework}

\title{Assignment 11}

\date{Monday April 23 at 6:00 PM}

\author{}

\newcommand{\reverse}{{\text{R}}}

\begin{document}

\begin{introduction}
  \theintroduction
\end{introduction}

\begin{enumerate}
\item \points{20} As shown in lecture, it is possible to construct a DFA with
  input alphabet \(\set{\mathtt 0, \mathtt 1}\) that decides the language of
  strings in \(\set{\mathtt 0, \mathtt 1}^\star\) that, when interpreted as
  binary numerals most-significant bit first, represent numbers divisible by 3.
  [As a convention, we will take the empty string \(\varepsilon\) to represent
  number \(0\) as well, even though technically it is not a binary numeral.]
  \begin{enumerate}
  \item Describe how for any \(n>0\), the set \(L_n\) of strings of
    \(\mathtt 0\)'s and \(\mathtt 1\)'s that are binary numerals of numbers
    divisible by \(n\) is regular (i.e. there is a DFA that decides it.)  [Hint:
    What is the effect on the modulus of appending an additional \(\mathtt 0\)
    vs. an additional \(\mathtt 1\)?]

  \item Illustrate your construction by showing the exact rules for \(n=7\).

  \item \label{part:mod-7-dfa} Run your \(\mathop{\bmod} 7\) machine on the
    simulator \texttt{dfa.pro} and show the output for all input strings of
    length 7.
  \end{enumerate}
  Note: It is \textbf{not} recommended that you use induction on \(n\) here.
  Instead, describe the state-transitions for any given \(n\) numerically, using
  arithmetic modulo \(n\).

\item \points{30} \label{problem:reverse} By the reverse \(x^\reverse\) of a string,
  we mean the string consisting of the same symbols as in \(x\), but in the
  reverse order.  For example, \(\mathtt{01011}^\reverse = \mathtt{11010}\).  By
  the reverse of a language \(L\), we mean the language consisting of the
  reverses of strings in \(L\):
  \[
    L^\reverse = \set{x^\reverse \mid x \in L}.
  \]
  \begin{enumerate}
  \item Show by construction that regular languages are closed under reversal,
    i.e. if \(L\) is regular, then so is \(L^\reverse\).

  \item If \(L\) represents numerals most-significant-bit first, then
    \(L^\reverse\) represents numerals least-significant-bit first.  Construct
    the reverse of the \(\mathop{\bmod} 3\) machine presented in the lecture
    notes.  What is interesting about this example?

  \item Construct the reverse of the \(\mathop{\bmod} 7\) machine from part
    \ref{part:mod-7-dfa}.  Show the output of the simulator \texttt{dfa.pro} for
    input strings of length 7.
  \end{enumerate}

\item \points{10} The reverse \(x^\reverse\) of a string was defined in problem
  \ref{problem:reverse}.  Show that if \(L\) is regular then so is
  \[
    L' = \set{x \in L \mid x^\reverse \in L}.
  \]
  \(L'\) is thus the subset of \(L\) such that for any string in the subset, the
  reverse of the string is also in \(L\).

  For example, if \(L = \set{\mathtt 0, \mathtt{01}}^\star\), then
  \(L' = \set \varepsilon \cup \set{\mathtt 0} \set{\mathtt 0,
    \mathtt{10}}^\star\), as
  \[
    \begin{array}{r @{\;=\{} l @{\,} l @{\,} l @{\,} l @{\,} l @{\,} l @{\,} l
        @{\,} l @{\,} l @{\,} l @{\,} l @{\,} l @{,\dots\},}}
      L & \varepsilon, &\mathtt 0, &\mathtt{00}, &\mathtt{01},
      &\mathtt{000}, &\mathtt{001}, &\mathtt{010}, &\mathtt{0000}, &\mathtt{0010},
      &\mathtt{0100}, &\mathtt{0101}, &\mathtt{00000} \\
      L' && \mathtt 0, &\mathtt{00}, &&\mathtt{000}, &&\mathtt{010},
      &\mathtt{0000}, &\mathtt{0010}, &\mathtt{0100}, &&\mathtt{00000}
    \end{array}
  \]
  where we've aligned the elements of each to show which are in the subset.

\item \points{10} Consider the language \(T = \set{\mathtt 1^t \mid \text{\(t\)
      is a triangular number}} = \set{\varepsilon, \mathtt 1, \mathtt{111},
    \mathtt{111111}, \mathtt{1111111111}, \dots}\)
  (\(0, 1, 3, 6, 10, 15, \dots\)).  Use the Myhill-Nerode theorem to show that
  \(T\) is not regular.

\item \points{10} Let \(\Sigma\) and \(\Delta\) be alphabets.  A \emph{string
    homomorphism} is a function \(h \colon \Sigma^\star \to \Delta^\star\) such
  that for any \(x \in \Sigma^\star\) there is a string
  \(h(x) \in \Delta^\star\) as follows:

  For single letters \(\sigma \in \Sigma\), \(h(x) \in \Delta^\star\) is
  specified as a \emph{base}.  Then for strings:
  \[
    h(\sigma_1 \sigma_2 \sigma_3 \dots \sigma_n) = h(\sigma_1) h(\sigma_2)
    h(\sigma_3) \dots h(\sigma_n), \quad \text{each \(\sigma_i \in \Sigma\)}.
  \]
  (For \(n = 0\), \(h(\varepsilon) = \varepsilon\).)

  For example, suppose \(\Sigma = \set{\mathtt a, \mathtt b}\) and
  \(\Delta = \set{\mathtt 0, \mathtt 1, \mathtt 2}\).  One string homomorphism
  is given by \(h(\mathtt a) = \mathtt{0112}, h(\mathtt b) = \varepsilon\).
  Then, for example,
  \[
    h(\mathtt{aba}) = \mathtt{0112}\varepsilon\mathtt{0112} = \mathtt{01120112}.
  \]

  A string homomorphism is applied to an entire language \(L\) as
  \[
    h(L) = \set{h(x) \mid x \in L}.
  \]

  Show that the regular languages are closed under string homomorphism, i.e. if
  \(L\) is regular, and \(h\) is a string homomorphism, then \(h(L)\) is
  regular.

  [Suggestion: Maybe use the idea of an NFA.]

\item \points{20} Let \(\Sigma\) and \(\Delta\) be alphabets.  A
  \emph{substitution} is a function \(h \colon \Sigma \to 2^{\Delta^\star}\)
  (where \(2^{\Delta^\star}\) is the set of all languages of \(\Delta\)).  That
  is, for each \(\sigma \in \Sigma\), \(h(\sigma) \subseteq \Delta^\star\), a
  language.  Then for strings,
  \[
    h(\sigma_1 \sigma_2 \sigma_3 \dots \sigma_n) = h(\sigma_1) h(\sigma_2)
    h(\sigma_3) \dots h(\sigma_n), \quad \text{each \(\sigma_i \in \Sigma\)},
  \]
  where the concatenation here is the concatenation of languages rather than
  strings.  (For \(n = 0\), \(h(\varepsilon) = \set{\varepsilon}\).)

  A substiution is applied to an entire language \(L\) as
  \[
    h(L) = \bigcup \set{h(x) \mid x \in L}.
  \]
  That is, to get \(h(L)\) for a language, we take the union of \(h\) applied to
  individual strings in the language.

  A \emph{regular} substitution is one where each \(h(\sigma)\) is a regular
  language.  Show that a regular substitution applied to a regular langauge
  results in a regular language.

  [Hint: Maybe use regular expressions.]
  
  
\end{enumerate}
\end{document}
  