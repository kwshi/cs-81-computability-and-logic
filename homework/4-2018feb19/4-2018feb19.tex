\documentclass{cs81-homework}

\author{}
\title{Assignment 4}
\date{2018 February 19 (Mondays)}

\usepackage{prftree}

\begin{document}

\begin{introduction}
  \theintroduction

  For tableau proofs, it suggested that you use the spreadsheet method. Please
  remember to turn off the gridlines from the View menu before taking your
  screenshot for greater readability.

  Once your document is complete, make a pdf and submit to Gradescope.

  Please help us grade by placing your solutions in the spaces provided.
\end{introduction}

\begin{enumerate}
\item \points{10} For the following entailment, use the tableau method to either
  prove that it is valid or construct a counterexample.
  \[
    \forall x \, (\exists y \, A(x, y) \to \forall z \, A(z, x)) \entails
    \forall x \, \forall y \, \forall z \, (A(x, y) \to A(z, x)).
  \]

  \begin{solution}
  \end{solution}

\item \points{10} For the following entailment, use the tableau method to either
  prove that it is valid or construct a counterexample. (Recall that
  \(\exists x \, \top\) asserts a non-empty domain, and may be used in the
  tableau to provide a constant.)
  \[
    \exists x \, \top \entails \exists x \, (A(x) \to \exists y \, A(y))
  \]

  \begin{solution}
  \end{solution}

\item \points{10} For the following entailment, use the tableau method to either
  prove that it is valid or construct a counterexample.
  \[
    \entails \forall x \, (A(x) \to \forall y \, A(y)).
  \]

  \begin{solution}
  \end{solution}

\item[\ref{itm:prenex-1} \& \ref{itm:prenex-2}.] Prenex Rules are used to
  justify movement of quantifiers in predicate logic formulas. These have great
  applicability in automated reasoning and theorem proving. You should read
  about the prenex rules here. In this problem, we want to justify only one of
  the four basic rules:
  \[
    A(x) \lor \forall y \, B(y) \implies \forall y \, (A(x) \lor B(y))
  \]
  where \(\implies\) means ``replace with'', and \(A(x)\) is a formula not
  having \(y\) as a free variable.  The justification for this rule involves
  proving the following two sequents.
    \begin{enumerate}
    \item[\ref{itm:prenex-1}.]
      \(A(x) \lor \forall y \, B(y) \derives \forall y \, (A(x) \lor B(y))\)
      
    \item[\ref{itm:prenex-2}.]
      \(\forall y \, (A(x) \lor B(y)) \derives A(x) \lor \forall y \, B(y)\)
    \end{enumerate}
    While the proof of \ref{itm:prenex-1} is straightforward, I found it helpful
    for \ref{itm:prenex-2} to use a tableau the first time I did it.

    \mbox{}

\item \label{itm:prenex-1} \points{5} Give a natural deduction proof of
\(A(x) \lor \forall y \, B(y) \derives \forall y \, (A(x) \lor B(y))\).

[Here \(x\) is a free variable on both sides. Treat it as if a constant.]

  \begin{solution}
  \end{solution}

\item \label{itm:prenex-2} \points{15} Give a natural deduction proof of
  \(\forall y \, (A(x) \lor B(y)) \derives A(x) \lor \forall y \, B(y)\).

  [Here \(x\) is a free variable on both sides. Treat it as if a constant.]

  \begin{solution}
  \end{solution}

\item \points{20} As we have seen,
  \[
    \exists x \, \forall y \, L(x, y) \derives \forall y \, \exists x \, L(x, y),
  \]
  but generally
  \[
    \forall y \, \exists x \, L(x, y) \not \derives \exists x \, \forall y \,
    L(x, y),
  \]
  even if we add a premise \(\exists z \, \top\) to ensure a non-empty domain.

  But for the special case where \(L(x, y)\) can be factored as
  \(A(x) \lor B(y)\), the second direction
  \(\exists z \, \top, \forall y \, \exists x \, (A(x) \lor B(y)) \derives
  \exists x \, \forall y (A(x) \lor B(y))\) is provable.

  Show that this is the case by giving a tableau proof.

  \begin{solution}
  \end{solution}

\item \points{10} Let \(f\) and \(g\) be \(1\)-ary function symbols. Any
  \(1\)-ary function \(\varphi\) is called one-to-one provided that the
  following formula is satisfied:
  \[
    \forall x \, \forall y \, ((\varphi(x) = \varphi(y)) \to (x = y)).
  \]
  Give a tableau proof of the following:
  \begin{center}
    If \(f\) and \(g\) are one-to-one, so is the composition of \(f\) with \(g\).
    In other words, prove this entailment
  \end{center}
  \begin{gather*}
    \forall x \, \forall y \, ((f(x) = f(y)) \to (x = y)), \\
    \forall x \, \forall y \, ((g(x) = g(y)) \to (x = y)) \\
    \entails \forall x \, \forall y \, ((f(g(x)) = f(g(y))) \to (x = y)).
  \end{gather*}

  \begin{solution}
  \end{solution}

\item \label{itm:group} \points{20} Recall these group axioms from the lecture
  notes, where \(*\) (group ``multiplication'') is a \(2\)-ary function symbol
  in infix notation and \(^{-1}\) (group inverse) is a \(1\)-ary function symbol
  in superscript notation, together with equality rules:
  \begin{center}
    \begin{tabular}{|l|l|l|}
      \hline
      Associative Rule & \multicolumn{2}{l|}
                         {GA: \(\forall x \, \forall y \, \forall z \,
                         (x*y)*z = x*(y*z)\)} \\ \hline
      Identity Rules & GLI: \(\forall x \, 1*x=x\) & GRI: \(\forall x \, x*1=x\) \\ \hline
      Inverse Rules & GLV: \(\forall x \, x^{-1} * x = 1\) & GRV: \(\forall x \, x * x^{-1} = 1\) \\ \hline
    \end{tabular}
  \end{center}
  Recall also the following rule for equality and functions:
  \begin{description}
  \item[\(=\) Introduction Rule]
    \[
      \prftree[r]{\quad (no antecedent)\quad where \(t\) is any term}{t=t}
    \]
    
  \item[\(=\) Elimination Rule]
    \[
      \prftree[r]{\quad where \(A(x)\) is any formula}{s=t}{A(s)}{A(t)}
    \]
    This allows replacing a term with another term equal to the first.

  \item[Function Equality Rule] (derivable from the \(=\) Elimination Rule)
    \[
      \prftree[r]{\quad where \(s_1, s_2, t_1, t_2\) are any terms}{s_1 =
        t_1}{s_2 = t_2}{s_1*s_2 = t_1*t_2}
    \]
  \end{description}
  Prove that the inverse function \(-1\) is one-to-one using natural deduction,
  i.e.
  \[
    \forall x \, \forall y \, \del{x^{-1} = y^{-1} \to x = y}.
  \]
  \textbf{Note}: The purpose of this problem is to exercise the \textbf{form} of
  a natural deduction proof using axioms and equality rules, rather than the
  result itself, which is fairly simple.  Try not to skip steps.  As JAPE
  natural deduction does not include these function symbols, maybe use a
  spreadsheet or \LaTeX\ for formatting.

  \newpage

\item[\ref{itm:group}.] \points{20} Proof.
  
  \begin{solution}
  \end{solution}

\end{enumerate}

\end{document}