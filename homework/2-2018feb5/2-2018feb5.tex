\documentclass{cs81-homework}

\title{Homework 2}
\author{}
\date{2018 February 5 (Monday)}

\begin{document}

\begin{introduction}
  
  \paragraph{Notice}
  When you submit this assignment, you are certifying therewith
  that you understand and accept the following policy, which applies to all
  assignments.

  \paragraph{Collaboration Policy}
  The writeup that you submit must be your own work. You are encouraged to get
  help from the professor and grutors. You may discuss the problems with
  classmates, but if you do so, it should be in groups of no more than three. You
  are not allowed to copy or transcribe solutions from other sources, including
  the work of other students, the internet, previous solution sets, and images
  photographed from a whiteboard or blackboard. There is to be no “partnering”
  where two or more students submit the same writeup. If you get help on a
  problem, you should say who provided the help on a per-problem basis. Blanket
  statements such as “worked with John and Mary” are not allowed. Detected
  infractions may impact your academic career.

  \paragraph{Formatting Policy}
  All work must be typeset in electronic media and submitted as a single pdf file,
  one problem on each page as shown in the following pages. Retain this header
  page. Handwritten and photographed or scanned work is not allowed.

  Do not use inverse video (light typography on dark background). Do not rotate
  images.  You will not get credit for difficult-to-read submissions.

  For JAPE proofs take a screenshot of your proof and paste it into the
  document. For written out proofs, just type into a copy of the Google doc master
  (not the pdf), or use some other method such as LaTex if you must. You may
  scrape formulas and symbols directly from the doc master, or from my page full
  of symbols that I use throughout the course. Once your document is complete,
  make a pdf and submitted to Gradescope.

\end{introduction}

\begin{enumerate}
\problem{5} Show that a constructive proof of \(\neg A \to A \derives A\)
  is not possible by deriving LEM (\(\derives B \lor \neg B\)) from it, using
  only constructive rules in addition.

  \begin{solution}
  \end{solution}

\problem{10} Use the tableau method to establish whether or not the
  following is valid. If not valid, derive a counterexample from the tableau.
  \[
    A \to (B \to \neg C), C \entails \neg A \lor \neg B
  \]

  \begin{solution}
  \end{solution}

\problem{10} Use the tableau method to establish whether or not the
  following is valid. If not valid, derive a counterexample from the tableau.
  \[
    (A \to B) \to C \entails A \to (B \to C)
  \]
  
  \begin{solution}
  \end{solution}

\problem{10} Determine whether the following is valid using a tableau. If
  not, extract a counterexample from the tableau.
  \[
    \entails A \lor \neg B \lor (\neg A \land \neg C) \lor (B \land \neg C) \lor
    (C \land D)
  \]

  \begin{solution}
  \end{solution}

\item[] In 5--7, use the method described in lecture (and approximately slides 343--363 of 
  “A Multi-Threaded Approach to Logic (Propositional Logic)”), to prove
  \[
    (A \to C) \land (B \to C) \derives (A \lor B) \to C.
  \]
  Note that this is intended to exercise a specific method.
  
\problem{10} First construct a tableau showing
  \[
    (A \to C) \land (B \to C) \entails (A \lor B) \to C.
  \]

  \begin{solution}
  \end{solution}

\problem{25} Having constructed the closed tableau in the previous problem,
  convert it using the prescribed method to a natural deduction proof of
  \[
    (A \to C) \land (B \to C) \derives (A \lor B) \to C.
  \]

  (Using JAPE, you will need to prove some of the five lemmas in order to carry
  out your proof, some of which you did on the previous assignment). Note that
  in JAPE, if you want to get a pair of \(\land\) eliminations to stack left to
  right for aesthetic reasons, you should choose “\(\land\) elimination
  preserving right” first. Also, you may need to select the conclusion inside a
  box, in addition to the antecedent, to get a hypothesis to be inside the box.)

  Include proofs of any lemmas or derived rules you used.  There should be an
  evident correspondence between the tableau of problem 5 and the natural
  deduction proof of problem 6.

  \begin{solution}
  \end{solution}

\problem{10} Show that there is a constructive proof of
  \[
    (A \to C) \land (B \to C) \derives (A \lor B) \to C.
  \]

  \begin{solution}
  \end{solution}

\problem{20} The Sequent Calculus is a method similar to the tableau
  method, but which does not explicitly negate the conclusion. It constructs a
  proof by working backward from the desired sequent, constructing a tree. One
  difference is that, in the sequent calculus, there can be multiple statements
  on both sides of the turnstile: On the left-hand side, statements have their
  usual conjunctive meaning.


  On the right-hand side, statements have a disjunctive meaning. For example,
  \(A \to B \derives \neg A, B\) means that from \(A \to B\) prove either
  \(\neg A\) or \(B\).  In the sequent calculus, one sequent is replaced with
  another, until we get to a point where the same statement exists on both sides
  of \(\derives\) (a closed path) or until we can make no further replacements
  (an open path).  In some cases, branching takes place. The original sequent is
  proved if all paths can be made to close.  The rules for sequent calculus deal
  with each connective as it occurs on the left and on the right. Here \(\Gamma\) and \(\Delta\)
  represent arbitrary sets of statements.

  \begin{description}
  \item[Closure Rule] \(\Gamma \cup \set A \derives \Delta \cup \set A\) closes \\

  \item[\(\land\) Left Rule] \(\Gamma \cup \set{A \land B} \derives \Delta\) is
    replaced with \(\Gamma \cup \set{A, B} \derives \Delta\)
    
  \item[\(\land\) Right Rule] \(\Gamma \derives \Delta \cup \set{A \land B}\)
    splits into \(\Gamma \derives \Delta \cup \set A\) and
    \(\Gamma \derives \Delta \cup \set B\) \\

  \item[\(\lor\) Left Rule] \(\Gamma \cup \set{A \lor B} \derives \Delta\) splits
    into \(\Gamma \cup \set A \derives \Delta\) and
    \(\Gamma \cup \set B \derives \Delta\)
    
  \item[\(\lor\) Right Rule] \(\Gamma \derives \Delta \cup \set{A \lor B}\) is
    replaced with \(\Gamma \derives \Delta \cup \set{A, B}\) \\

  \item[\(\to\) Left Rule] \(\Gamma \cup \set{A \to B} \derives \Delta\) splits
    into \(\Gamma \derives \Delta \cup \set A\) and
    \(\Gamma \cup \set B \derives \Delta\)
    
  \item[\(\to\) Right Rule] \(\Gamma \derives \Delta \cup \set{A \to B}\) is replaced
    with \(\Gamma \cup \set A \derives \Delta \cup \set B\) \\

  \item[\(\neg\) Left Rule] \(\Gamma \cup \set{\neg A} \derives \Delta\) is
    replaced \(\Gamma \derives \Delta \cup \set A\)
    
  \item[\(\neg\) Right Rule] \(\Gamma \derives \Delta \cup \set{\neg A}\) is
    replaced with \(\Gamma \cup \set A \derives \Delta\)
  \end{description}

  Here is an example of a sequent calculus derivation:
  \begin{center}
    \begin{prooftree}
      {}
      [\lnot (A \lor B) \derives (\lnot A \land \lnot B), just=Goal
      [{\derives (A \lor B), (\lnot A \land \lnot B)}, just=\(\lnot\) Left Rule:!u
      [{\derives A, B, (\lnot A \land \lnot B)}, just=\(\lor\) Right Rule:!u
      [{\derives A, B, \lnot A}, just=\(\land\) Right Rule:!u
      [{A \derives A, B}, just=\(\lnot\) Right:!u, close]]
      [{\derives A, B, \lnot B}
      [{B \derives A, B}, just=\(\lnot\) Right:!u, close]]]]]
    \end{prooftree}
  \end{center}

  The problem is to construct a sequent calculus proof of
  \[
    (A \to C) \land (B \to C) \derives (A \lor B) \to C.
  \]

  \begin{solution}
  \end{solution}

\end{enumerate}

\end{document} 