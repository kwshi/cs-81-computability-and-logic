\documentclass{cs81-homework}

\title{Assignment 10}

\date{Monday April 16 at 6:00 PM}

\author{}

\begin{document}

\begin{introduction}
  \theintroduction
\end{introduction}

\begin{enumerate}
\item \points{20} Two Turing Machines \(M\) and \(N\) are equivalent iff they
  recognize the same language, i.e. \(L(M) = L(N)\).  Show that the language
  \[
    \set{\abr{M, N} \mid \text{\(M\) is equivalent to \(N\)}}
  \]
  is neither recognizable nor corecognizable.

  [Hint: Choose \(N\) appropriately for each case (recognizable, corecognizable)
  and reduce a known unrecognizable problem to the case in question.]

  \begin{solution}
  \end{solution}

\item \points{10} Show that a language \(L\) is decidable iff it is
  mapping-reducible to some finite language.

  \begin{solution}
  \end{solution}

\item \points{10} Show that if \(L\) is a recognizable language, and
  \(L \le_m L^\complement\) (i.e. \(L\) is \emph{mapping reducible} to its own
  complement) then \(L\) is decidable.

  \begin{solution}
  \end{solution}

\item \points{40} Classify each of the language below into one of the
  categories, and state your reason for the classification:
  \begin{enumerate}[label=\Roman*.]
  \item Unrecognizable by Rice's Theorem
  \item Uncorecognizable by Rice's Theorem
  \item Rice's theorem doesn't apply, but it is decidable
  \item Rice's theorem doesn't apply, and it is not decidable
  \end{enumerate}
  The languages are (for brevity, we omit ``\(M\) is a Turing machine such
  that'' in each case):
  \begin{align*}
    A &= \set{\abr M \mid \text{\(M\) accepts more than 81 strings}}, \\
    B &= \set{\abr M \mid \text{\(M\) accepts not more than 81 strings}}, \\
    C &= \set{\abr M \mid \text{\(M\) has no rejecting control states}}, \\
    D &= \set{\abr M \mid \text{\(M\) uses no more than 81 steps when started on blank tape}}, \\
    E &= \set{\abr M \mid \text{\(M\) \(\exists x \in \Sigma^\star\),
        \(M\) uses more than 81 steps when started on \(x\)}}, \\
    F &= \set{\abr M \mid \text{\(L(M)\) is decidable by some finite-state machine (DFA)}}, \\
  \end{align*}
  As usual, \(L(M)\) is the language recognized by Turing machine \(M\).

  \begin{solution}
  \end{solution}

\item \points{20} Demonstrate by proof of Church's Theorem by converting your
  Turing machine rules for \texttt{tri1} from a08 into Prover9 clauses for
  resolution and running it with input tape \texttt{[1, 1, 1, 1]} in Prover9.
  It is expected that the tape will contain \texttt{[1, 1, 1, 1, 1, 1, 1, 1, 1,
    1]} (10 \texttt{1}s) when it halts.  Show a second run with a ``damaged''
  version of the machine that does not halt at all.  (If you didn't have a
  correct solution for \texttt{tri1}, you may use the one in the provided
  solutions instead.)

  You may copy the basic simulation clauses from this handout.  You will need to
  make a few changes:
  \begin{enumerate}[label=\alph*.]
  \item The blank symbol on my rules is set up for \texttt{0} as the blank
    symbol.  You might have used a different symbol for blank, in which case you
    will have to either edit your rules or mine accordingly.

  \item Instead of listing the rules in brackets as in \texttt{tm.pro}, each
    rule is a separate clause, of the form \texttt{m(xControl, xRead, yWrite,
      yControl, yMove).}  ending with a period.  The rule clauses below give an
    example.

  \item You should not be using any state or tape symbols that begin with the
    letters \(u, v, w, x, y, z\) as these are recognized as variables in
    Prover9, and will cause unexpected results.

    \lstinputlisting{prover9/tm.prover9}
  \end{enumerate}
  
  \begin{solution}
  \end{solution}

\end{enumerate}

\end{document}